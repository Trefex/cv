\documentclass[11pt,a4paper,nolmodern]{moderncv}        % possible options include font size ('10pt', '11pt' and '12pt'), paper size ('a4paper', 'letterpaper', 'a5paper', 'legalpaper', 'executivepaper' and 'landscape') and font family ('sans' and 'roman')

\usepackage[firstyear=1996,lastyear=2018]{moderntimeline}
%\tlmaxdates{1999}{2012}
\tlwidth{0.8ex}
\usepackage{xpatch}
%\usepackage[style=authoryear,sorting=ydnt,dashed=false]{biblatex}
\usepackage[url=false,backend=biber,
	style=authoryear,
	doi=false,  
	isbn=false,
	backref=false, 
	%maxnames=1,
	%minnames=1,
	maxbibnames=6,
	minbibnames=6,
	%uniquename=false, 
	sorting=ydnt]{biblatex}

%\renewbibmacro*{date}{}
%\renewbibmacro*{date+extrayear}{}
%\renewbibmacro*{issue+date}{}
%\newcommand*{\bibyear}{}
%
%\defbibenvironment{bibliography}
  %{\list
     %{\iffieldequals{year}{\bibyear}
        %{}
        %{\tldatelabelcventry{\printfield{year}}{Jul.-Aug. 2000}{}{}{}{}{}%
         %%\savefield{year}{\bibyear}
				%}}
     %{\setlength{\topsep}{0pt}% layout parameters based on moderncvstyleclassic.sty
      %\setlength{\labelwidth}{\hintscolumnwidth}%
      %\setlength{\labelsep}{\separatorcolumnwidth}%
      %\setlength{\itemsep}{\bibitemsep}%
      %\leftmargin\labelwidth%
      %\advance\leftmargin\labelsep}%
      %\sloppy\clubpenalty4000\widowpenalty4000}
  %{\endlist}
  %{\item}
	
	%remove brackets from year
\xpatchbibmacro{date+extrayear}{%
  \printtext[parens]}{\printtext}{}{}

% remove year from the author bibmacro
\xpatchbibmacro{author}{%
 \usebibmacro{date+extrayear}}
 {}{}{}

%change order and wrap into \cvline
\DeclareBibliographyDriver{article}{%
\tldatelabelcventry{
	%\usebibmacro{bibindex}%
  %\usebibmacro{begentry}%
	\thefield{year} % Year 
	}{
	\thefield{year} % Label of year
	}{
	\usebibmacro{bibindex}%
  \usebibmacro{begentry}%
  \usebibmacro{author/translator+others}}{}{%
	\usebibmacro{bibindex}%
  \usebibmacro{begentry}%
  %\usebibmacro{author/translator+others}
  %
  \setunit{\labelnamepunct}\newblock
  \usebibmacro{title}%
  \newunit
  \printlist{language}%
  \newunit\newblock
  \usebibmacro{byauthor}%
  \newunit\newblock
  \usebibmacro{bytranslator+others}%
  \newunit\newblock
  \printfield{version}%
  \newunit\newblock
  \usebibmacro{in:}%
  \usebibmacro{journal+issuetitle}%
  \newunit
  \usebibmacro{byeditor+others}%
  \newunit
  \usebibmacro{note+pages}%
  \newunit\newblock
  \iftoggle{bbx:isbn}
    {\printfield{issn}}
    {}%
  \newunit\newblock
  \usebibmacro{addendum+pubstate}%
  \setunit{\bibpagerefpunct}\newblock
  \usebibmacro{pageref}%
  \newunit\newblock
 \usebibmacro{related}%
  \usebibmacro{finentry}
	}%
  }
	
	
	\DeclareBibliographyDriver{inproceedings}{%
\tldatelabelcventry{
	%\usebibmacro{bibindex}%
  %\usebibmacro{begentry}%
	\thefield{year} % Year 
	}{
	\thefield{year} % Label of year
	}{
	%\usebibmacro{bibindex}%
  %\usebibmacro{begentry}%
  \usebibmacro{author/translator+others}}{%
	\usebibmacro{bibindex}%
  \usebibmacro{begentry}%
  %\usebibmacro{author/translator+others}
  %
  \setunit{\labelnamepunct}\newblock
  \usebibmacro{title}%
  \newunit
  \printlist{language}%
  \newunit\newblock
  \usebibmacro{byauthor}%
  \newunit\newblock
  \usebibmacro{bytranslator+others}%
  \newunit\newblock
  \printfield{version}%
  \newunit\newblock
  \usebibmacro{in:}%
  \usebibmacro{booktitle}%
  \newunit
  \usebibmacro{byeditor+others}%
  \newunit
	%\thefield{pages}%
  \newunit\newblock
  \iftoggle{bbx:isbn}
    {\printfield{issn}}
    {}%
  \newunit\newblock
  \usebibmacro{addendum+pubstate}%
  \setunit{\bibpagerefpunct}\newblock
  \usebibmacro{pageref}%
  \newunit\newblock
  \usebibmacro{related}%
  \usebibmacro{finentry}}%
  }
	%\DeclareBibliographyDriver{inproceedings}{%
%\tldatelabelcventry{\usebibmacro{date+extrayear}}{label}{}{%
  %\usebibmacro{bibindex}%
  %\usebibmacro{begentry}%
  %\usebibmacro{author/translator+others}%
  %\setunit{\labelnamepunct}\newblock
  %\usebibmacro{title}%
  %\newunit
  %\printlist{language}%
  %\newunit\newblock
  %\usebibmacro{byauthor}%
  %\newunit\newblock
  %\usebibmacro{bytranslator+others}%
  %\newunit\newblock
  %\printfield{version}%
  %\newunit\newblock
  %\usebibmacro{in:}%
  %\usebibmacro{journal+issuetitle}%
  %\newunit
  %\usebibmacro{byeditor+others}%
  %\newunit
  %\usebibmacro{note+pages}%
  %\newunit\newblock
  %\iftoggle{bbx:isbn}
    %{\printfield{issn}}
    %{}%
  %\newunit\newblock
  %\usebibmacro{doi+eprint+url}%
  %\newunit\newblock
  %\usebibmacro{addendum+pubstate}%
  %\setunit{\bibpagerefpunct}\newblock
  %\usebibmacro{pageref}%
  %\newunit\newblock
  %\usebibmacro{related}%
  %\usebibmacro{finentry}}%
  %}


\defbibheading{bibliography}[Publications]{\section{#1}}

% moderncv themes
\moderncvstyle{classic}                             % style options are 'casual' (default), 'classic', 'oldstyle' and 'banking'
\moderncvcolor{blue}                               % color options 'blue' (default), 'orange', 'green', 'red', 'purple', 'grey' and 'black'
\renewcommand{\familydefault}{\sfdefault}         % to set the default font; use '\sfdefault' for the default sans serif font, '\rmdefault' for the default roman one, or any tex font name

%\nopagenumbers{}                                  % uncomment to suppress automatic page numbering for CVs longer than one page

% character encoding
\usepackage[utf8]{inputenc}                       % if you are not using xelatex ou lualatex, replace by the encoding you are using
%\usepackage{CJKutf8}                              % if you need to use CJK to typeset your resume in Chinese, Japanese or Korean

% adjust the page margins
\usepackage[scale=0.8]{geometry}
\setlength{\hintscolumnwidth}{3.5cm}                % if you want to change the width of the column with the dates
%\setlength{\makecvtitlenamewidth}{10cm}           % for the 'classic' style, if you want to force the width allocated to your name and avoid line breaks. be careful though, the length is normally calculated to avoid any overlap with your personal info; use this at your own typographical risks...

% personal data
\name{Christophe}{Trefois}
\title{Ing. sys. com. dipl. EPF}                               % optional, remove / comment the line if not wanted
\address{38, rue de Kleinbettingen}{L-8436 Steinfort}{}% optional, remove / comment the line if not wanted; the "postcode city" and "country" arguments can be omitted or provided empty
\phone[mobile]{+352~661~777~175}                   % optional, remove / comment the line if not wanted; the optional "type" of the phone can be "mobile" (default), "fixed" or "fax"
%\phone[fixed]{+2~(345)~678~901}
%\phone[fax]{+3~(456)~789~012}
\email{christophe@trefois.com}                               % optional, remove / comment the line if not wanted
\homepage{www.trefois.com}                         % optional, remove / comment the line if not wanted
\social[linkedin]{trefoischristophe}                        % optional, remove / comment the line if not wanted
%\social[twitter]{Trefex}                             % optional, remove / comment the line if not wanted
\social[github]{Trefex}                              % optional, remove / comment the line if not wanted
%\extrainfo{additional information}                 % optional, remove / comment the line if not wanted
\photo[68pt][0.4pt]{cv_cts_cropped.png}                       % optional, remove / comment the line if not wanted; '64pt' is the height the picture must be resized to, 0.4pt is the thickness of the frame around it (put it to 0pt for no frame) and 'picture' is the name of the picture file
%\quote{Some quote}                                 % optional, remove / comment the line if not wanted

% to show numerical labels in the bibliography (default is to show no labels); only useful if you make citations in your resume
%\makeatletter
%\renewcommand*{\bibliographyitemlabel}{\@biblabel{\arabic{enumiv}}}
%\makeatother
%\renewcommand*{\bibliographyitemlabel}{[\arabic{enumiv}]}% CONSIDER REPLACING THE ABOVE BY THIS

% bibliography with mutiple entries
%\usepackage{multibib}
%\newcites{book,misc}{{Books},{Others}}

\usepackage[originalcommands]{ragged2e}
\renewcommand*{\cvcomputer}[4]{%
  \cvdoubleitem{#1}{\small\raggedright#2}{#3}{\small\raggedright#4}}
% Not all fonts have an sl shape
\renewcommand*{\cventry}[6]{%
  \cvline{#1}{%
    {\bfseries#2}%
    \ifx#3\else{, {\itshape#3}}\fi%
    \ifx#4\else{, #4}\fi%
    \ifx#5\else{, #5}\fi%
    %.%
    \ifx#6\else{\newline{}\begin{minipage}[t]{\linewidth}\small#6\end{minipage}}\fi
    }}%
		
		

\addbibresource{publications.bib}

%\xpatchcmd\cventry{.\strut}{\strut}{}{}
%----------------------------------------------------------------------------------
%            content
%----------------------------------------------------------------------------------
\begin{document}
%\begin{CJK*}{UTF8}{gbsn}                          % to typeset your resume in Chinese using CJK
%-----       resume       ---------------------------------------------------------
\makecvtitle

\section{Education}
\tlcventry{2010}{0}{PhD candidate in Systems Biomedicine}{Luxembourg Centre for Systems Biomedicine (LCSB) - University of Luxembourg}{Luxembourg}{}{} %{Description}  % arguments 3 to 6 can be left empty
\tllabelcventry{2006}{2008}{2006--2008}{Master of Science in Communication Systems}{Ecole Polytechnique Fédérale de Lausanne (EPFL)}{Lausanne}{$5.32/6$}{} %{Description}  % arguments 3 to 6 can be left empty
\tllabelcventry{2003}{2006}{2003--2006}{Bachelor of Science in Communication Systems}{Ecole Polytechnique Fédérale de Lausanne (EPFL)}{Lausanne}{}{}
\tllabelcventry{1996}{2003}{1996--2003}{Baccalauréat scientifique (Section B)}{Lycée Athénée de Luxembourg}{Luxembourg}{\textit{cum laude}}{}

\section{Master thesis}
\cvitem{title}{\emph{Title}}
\cvitem{supervisors}{Supervisors}
\cvitem{description}{Short thesis abstract}

\section{Experience}
\subsection{Vocational}
\cventry{year--year}{Job title}{Employer}{City}{}{General description no longer than 1--2 lines.\newline{}%
Detailed achievements:%
\begin{itemize}%
\item Achievement 1;
\item Achievement 2, with sub-achievements:
  \begin{itemize}%
  \item Sub-achievement (a);
  \item Sub-achievement (b), with sub-sub-achievements (don't do this!);
    \begin{itemize}
    \item Sub-sub-achievement i;
    \item Sub-sub-achievement ii;
    \item Sub-sub-achievement iii;
    \end{itemize}
  \item Sub-achievement (c);
  \end{itemize}
\item Achievement 3.
\end{itemize}}
\tllabelcventry{2008}{2010}{2008--2010}{Senior IT Auditor and Consultant}{Ernst \& Young}{Luxembourg}{}
	{Description 1\newline{} Description line 2}
%\subsection{Miscellaneous}
\subsection{Teaching}
\tldatelabelcventry{2007}{Mar.--Jul. 2007}{Teaching assistant}{EPFL}{Switzerland}{}
		{Information and technology project course -- Bachelor in communication systems. Helping students on their assignments and supervising groups of 2 students for their java programming project.}
	\tllabelcventry{2006}{2007}{2006--2007}{Teaching assistant}{EPFL}{Switzerland}{}
		{Computer networking course -- Bachelor in communication systems. Helping students on their assignments and projects and overseeing exams.}
	\tldatelabelcventry{2006}{Mar.-Jul. 2006}{Teaching assistant}{EPFL}{Switzerland}{}
		{Java object oriented programming course -- Bachelor in communication systems. Helping students on their assignments and projects and overseeing and correcting exams.}
%\cventry{Jul. 1999--Aug. 1999}{Job title}{Employer}{City}{}{Description}
\tldatelabelcventry{2000}{Jul.-Aug. 2000}{Fund support}{State Street Bank}{Luxembourg}{}{Distribution of mail, sorting of print outs from various funds}{}
\tldatelabelcventry{1999}{Jul.-Aug. 1999}{Fund support}{State Street Bank}{Luxembourg}{}{Distribution of mail, sorting of print outs from various funds}{}

\section{Languages}
\cvitemwithcomment{Language 1}{Skill level}{Comment}
\cvitemwithcomment{Language 2}{Skill level}{Comment}
\cvitemwithcomment{Language 3}{Skill level}{Comment}

\section{Computer skills}
\cvdoubleitem{category 1}{XXX, YYY, ZZZ}{category 4}{XXX, YYY, ZZZ}
\cvdoubleitem{category 2}{XXX, YYY, ZZZ}{category 5}{XXX, YYY, ZZZ}
\cvdoubleitem{category 3}{XXX, YYY, ZZZ}{category 6}{XXX, YYY, ZZZ}

\section{Interests}
\cvitem{hobby 1}{Description}
\cvitem{hobby 2}{Description}
\cvitem{hobby 3}{Description}

\section{Extra 1}
\cvlistitem{Item 1}
\cvlistitem{Item 2}
\cvlistitem{Item 3. This item is particularly long and therefore normally spans over several lines. Did you notice the indentation when the line wraps?}

\section{Extra 2}
\cvlistdoubleitem{Item 1}{Item 4}
\cvlistdoubleitem{Item 2}{Item 5}
\cvlistdoubleitem{Item 3}{Item 6. Like item 3 in the single column list before, this item is particularly long to wrap over several lines.}

\section{References}
\begin{cvcolumns}
  \cvcolumn{Category 1}{\begin{itemize}\item Person 1\item Person 2\item Person 3\end{itemize}}
  \cvcolumn{Category 2}{Amongst others:\begin{itemize}\item Person 1, and\item Person 2\end{itemize}(more upon request)}
  \cvcolumn[0.5]{All the rest \& some more}{\textit{That} person, and \textbf{those} also (all available upon request).}
\end{cvcolumns}

% Publications from a BibTeX file without multibib
%  for numerical labels: \renewcommand{\bibliographyitemlabel}{\@biblabel{\arabic{enumiv}}}% CONSIDER MERGING WITH PREAMBLE PART
%  to redefine the heading string ("Publications"): \renewcommand{\refname}{Articles}
\nocite{*}
%\bibliographystyle{plain}
%\bibliography{publications}                        % 'publications' is the name of a BibTeX file
\printbibliography[title={Publications}]
% Publications from a BibTeX file using the multibib package
%\section{Publications}
%\nocitebook{book1,book2}
%\bibliographystylebook{plain}
%\bibliographybook{publications}                   % 'publications' is the name of a BibTeX file
%\nocitemisc{misc1,misc2,misc3}
%\bibliographystylemisc{plain}
%\bibliographymisc{publications}                   % 'publications' is the name of a BibTeX file

\clearpage
%-----       letter       ---------------------------------------------------------
% recipient data
\recipient{Company Recruitment team}{Company, Inc.\\123 somestreet\\some city}
\date{January 01, 1984}
\opening{Dear Sir or Madam,}
\closing{Yours faithfully,}
\enclosure[Attached]{curriculum vit\ae{}}          % use an optional argument to use a string other than "Enclosure", or redefine \enclname
\makelettertitle

Lorem ipsum dolor sit amet, consectetur adipiscing elit. Duis ullamcorper neque sit amet lectus facilisis sed luctus nisl iaculis. Vivamus at neque arcu, sed tempor quam. Curabitur pharetra tincidunt tincidunt. Morbi volutpat feugiat mauris, quis tempor neque vehicula volutpat. Duis tristique justo vel massa fermentum accumsan. Mauris ante elit, feugiat vestibulum tempor eget, eleifend ac ipsum. Donec scelerisque lobortis ipsum eu vestibulum. Pellentesque vel massa at felis accumsan rhoncus.

Suspendisse commodo, massa eu congue tincidunt, elit mauris pellentesque orci, cursus tempor odio nisl euismod augue. Aliquam adipiscing nibh ut odio sodales et pulvinar tortor laoreet. Mauris a accumsan ligula. Class aptent taciti sociosqu ad litora torquent per conubia nostra, per inceptos himenaeos. Suspendisse vulputate sem vehicula ipsum varius nec tempus dui dapibus. Phasellus et est urna, ut auctor erat. Sed tincidunt odio id odio aliquam mattis. Donec sapien nulla, feugiat eget adipiscing sit amet, lacinia ut dolor. Phasellus tincidunt, leo a fringilla consectetur, felis diam aliquam urna, vitae aliquet lectus orci nec velit. Vivamus dapibus varius blandit.

Duis sit amet magna ante, at sodales diam. Aenean consectetur porta risus et sagittis. Ut interdum, enim varius pellentesque tincidunt, magna libero sodales tortor, ut fermentum nunc metus a ante. Vivamus odio leo, tincidunt eu luctus ut, sollicitudin sit amet metus. Nunc sed orci lectus. Ut sodales magna sed velit volutpat sit amet pulvinar diam venenatis.

Albert Einstein discovered that $e=mc^2$ in 1905.

\[ e=\lim_{n \to \infty} \left(1+\frac{1}{n}\right)^n \]

\makeletterclosing

%\clearpage\end{CJK*}                              % if you are typesetting your resume in Chinese using CJK; the \clearpage is required for fancyhdr to work correctly with CJK, though it kills the page numbering by making \lastpage undefined
\end{document}


%% end of file `template.tex'.
